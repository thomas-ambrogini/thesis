%!TEX root = ../thesis.tex

% *****************************************************************************
% ******************************** CONCLUSION *********************************
% *****************************************************************************

\chapter*{Conclusion}

In conclusion, the primary objective of creating an effective communication framework between heterogeneous processors in embedded systems has been achieved with satisfying results.
The thesis has presented a comprehensive exploration of the design, implementation, and performance analysis of a pub/sub middleware for embedded systems.
The objective 

The evaluation of the pub/sub middleware has revealed highly promising results, affirming its potential advantages and practical viability. Notably, the middleware exhibited superior performance in resource-constrained environments, demonstrating capabilities such as impressive scalability and adaptability to diverse workloads. Its low-latency characteristics make it particularly well-suited for applications requiring real-time responsiveness.

Moreover, the middleware showcased robustness in dynamic environments, effectively adapting to changes in network topology and device dynamics. The evaluation emphasized the system's efficiency in handling real-time data, optimizing bandwidth usage, and maintaining a minimal memory footprint, essential features for resource-limited embedded environments. The observed advantages underscore its reliability and suitability for a wide range of embedded applications.

In addition to its performance attributes, the middleware's design lends itself to enhancing safety measures through the duplication of brokers responsible for data storage. This redundancy introduces a layer of fault tolerance, minimizing the risk of data loss or system failures. The dynamism of this approach ensures that the system can dynamically adapt to changes in the environment, providing a robust and fail-safe mechanism in safety-critical applications.

Looking ahead, potential future directions include integration with edge computing paradigms, enhancements in security and reliability mechanisms, exploration of artificial intelligence integration, and the introduction of new protocols.