%!TEX root = ../thesis.tex

% *****************************************************************************
% ********************************** CHAPTER 4 ********************************
% *****************************************************************************

\chapter{Implementation}

\section{I don't know}

\subsection{Marshalling and Unmarshalling messages}

\section{Results}

\subsection{Direct Communication}

In this section there are the measurements obtained by utilizing RPMsg
directly.
In their implementation every packet has 16 bytes of header, so the real
message can be MAX\_MSG\_DIMENSION - 16 bytes of dimension.

\subsubsection{RTOS Cores}

The round trip time for R5s processors is copmuted by timing a message exchange
between two different R5s.

The round trip time for M4 processor is computed by timing a message exchange
between the M4 core and one of the R5s.

\begin{table}
\centering
\caption{Round trip time results using RP Message between RTOS cores}
\label{table:direct_communication_RTOS_cores}
\begin{tabular}{lcccccc}
\toprule
Message Size (in bytes) & 32 & 64 & 128 & 256 & 512 & 1024 \\
\midrule
RTT R5 (in us) & 16.7 & 21.9 & 32.9 & 54.9 & 98.9 & 186.8 \\
RTT M4 (in us) & 35.7 & 48.2 & 73.2 & 125.7 & 228.1 & 435.7 \\
\bottomrule
\end{tabular}
\end{table}

\subsubsection{Linux and RTOS cores}

\begin{table}
\centering
\caption{Round trip time results using RP Message between Linux and RTOS cores}
\label{table:direct_communication_linux_RTOS_cores}
\begin{tabular}{lccccc}
\toprule
Message Size (in bytes) & 32 & 64 & 128 & 256 & 512 \\
\midrule
Linux and R5 (in us) & 50 & 61 & 77 & 108 & 168 \\
Linux and M4 (in us) & 64 & 80 & 110 & 177 & 306 \\
\bottomrule
\end{tabular}
\end{table}

\subsection{Broker Communication}

In this section there are the measurements obtained by using the pub/sub
middleware. There are two modes which are tested:

\begin{itemize}
    \item   Push: after that a producer sends a message to a brokre, the broker
            sends it directly to all the subscribed consumers.
    \item   Pull: the consumers have to ask for the message to be delivered.
\end{itemize}

\subsubsection{RTOS cores}

In this section only RTOS cores are tested (M4 and R5).

\subsubsection{Push}

When there is only one consumer and one producer in push mode the time needed
for the consumer and the producer are basically the same.

\begin{table}
\centering
\caption{Round trip time results using the middleware between RTOS cores in
         push mode}
\label{table:broker_communication_RTOS_cores_push}
\begin{tabular}{lcccccc}
\toprule
Message Size (in bytes) & 32 & 64 & 128 & 256 & 512 & 1024 \\
\midrule
Producer R5 Time (in us) & 73.9 & 90.4 & 121.7 & 182.7 & 317.4 & 501.1 \\
Producer M4 Time (in us) & 99.7 & 111.1 & 137.6 & 190.7 & 322.4 & 506.9 \\
Consumer R5 Time (in us) & 76.9 & 92.9 & 122.9 & 182.9 & 317.6 & 503.2 \\
Consumer M4 Time (in us) & 74.9 & 82.1 & 120.6 & 182.8 & 315.2 & 570.3 \\
\bottomrule
\end{tabular}
\end{table}

As it can be seen in the table \ref{table:broker_communication_RTOS_cores_push},
the time for the consumer on the M4 is the same for the consumer on the R5 core,
since no messages have to be sent from the consumer when the communication is
in push mode.

\begin{table}
\centering
\caption{Round trip time results using the middleware between RTOS cores with
         multiple consumers in push mode}
\label{table:broker_communication_RTOS_cores_multiple_consumers_push}
\begin{tabular}{lcccccc}
\toprule
Message Size (in bytes) & 32 & 64 & 128 & 256 & 512 & 1024 \\
\midrule
Producer Time (in us) & 277.2 & 300.9 & 358.9 & 478.2 & 744.9 & 1110.1 \\
Consumer 1 Time (in us) & 282.2 & 306.1 & 364.1 & 483.2 & 750.1 & 1115.1 \\
Consumer 2 Time (in us) & 763.4 & 765.4 & 804.1 & 991.4 & 1279.1 & 1503.4 \\
Consumer 3 Time (in us) & 539.4 & 510.1 & 634.6 & 754.2 & 1063.6 & 1283.4 \\
\bottomrule
\end{tabular}
\end{table}

The variations in time measurements among different consumers, which cab be
seen in \ref{table:broker_communication_RTOS_cores_multiple_consumers_push}
can be attributed to the system's booting sequence.
During the boot process, the initialiation of memory areas can differ,
consequently leading to variations in the prioritization of message delivery.
It becomes apparent that the order in which processes are booted significantly
influences the subsequent beavior of the system, resulting in different timing
outcomes across consumers.

\subsubsection{Pull}

\begin{table}
\centering
\caption{Round trip time results using the middleware between RTOS cores in
         pull mode}
\label{table:broker_communication_RTOS_cores}
\begin{tabular}{lcccccc}
\toprule
Message Size (in bytes) & 32 & 64 & 128 & 256 & 512 & 1024 \\
\midrule
Producer R5 Time (in us) & 93.1 & 104.1 & 121.2 & 152.4 & 212.7 & 301.8 \\
Producer M4 Time (in us) & 171.1 & 182.4 & 191.5 & 262.4 & 391.9 & 579.2 \\
Consumer R5 Time (in us) & 228.6 & 238.9 & 276.1 & 325.4 & 439.9 & 618.9 \\
Consumer M4 Time (in us) & 372.3 & 411.5 & 490.4 & 589.9 & 843.8 & 1236.7 \\
\bottomrule
\end{tabular}
\end{table}

\subsection{Linux on Cortex-A53}

In the Liux IPC library communication the maximum dimension for messages is
512 bytes.

\subsubsection{Push}

\begin{table}
\centering
\caption{Round trip time results using the middleware betewen Linux and RTOS
    cores in push mode}
\label{table:broker_communication_linux_RTOS_cores_push}
\begin{tabular}{lccccc}
\toprule
Message Size (in bytes) & 32 & 64 & 128 & 256 & 512 \\
\midrule
Producer Time(in us) & 98 & 127 & 191 & 264 & 393 \\
Consumer Time (in us) & 97 & 127 & 191 & 264 & 393 \\
\bottomrule
\end{tabular}
\end{table}

\subsubsection{Pull}

\begin{table}
\centering
\caption{Round trip time results using the middleware betewen Linux and RTOS
    cores in pull mode}
\label{table:broker_communication_linux_RTOS_cores_pull}
\begin{tabular}{lccccc}
\toprule
Message Size (in bytes) & 32 & 64 & 128 & 256 & 512 \\
\midrule
Producer Time(in us) & 148 & 171 & 227 & 231 & 367 \\
Consumer Time (in us) & 264 & 292 & 322 & 406 & 565 \\
\bottomrule
\end{tabular}
\end{table}
